Вагон массы $9216$ кг приходит в движение вследствие действия ветра,
дующего вдоль полотна, и движется по горизонтальному пути. Сопротивление
движению вагона равно $1/200$ его веса. Сила давления ветра
$P=kSu^2$, где $S$ --- площадь задней стенки вагона, подверженной
давлению ветра, равная $6$ м$^2$, $u$ --- скорость ветра относительно
вагона, а $k=1.2$. Абсолютная скорость ветра $v=12$ м/с. Считая
начальную скорость вагона равной нулю, определить:
$1)$ наибольшую скорость $v_{max}$ вагона;
$2)$ время $T$, которое потребовалось бы для достижения этой скорости;
$3)$ на каком расстоянии $x$ вагон наберет скорость $3$ м/с.
