$\omega=\sqrt{n^2_{1}+n^2{2}+2n_{1}{2}\cos\theta_{0}}$; неподвижный аксоид --- круговой конус
$\xi^2+\eta^2-\frac{n^2_{2}\sin^2\theta_{0}}{\left(n_{2}\sin\theta_{0}+n_{1}\right)^2}\zeta^2=0$ с осью $\zeta$ и углом раствора
$2\arcsin\frac{n_{2}\sin\theta_{0}}{\omega}$; подвижный аксоид --- круговой конус 
$x^2+y^2-\frac{n^2_{1}\sin\theta_{0}}{\left(n_{1}\cos\theta_{0}+n_{2}\right)^2}z^2=0$ с осью $z$ и углом раствора $2\arcsin\frac{n_{1}\sin\theta_{0}}{\omega}$.
