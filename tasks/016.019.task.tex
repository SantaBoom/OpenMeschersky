Стержни $O_{1}A$ и $O_{2}B$, соединенные со стержнем $AB$ посредством шарниров $A$ и $B$, могут вращаться
вокруг неподвижных точек $O_{1}$ и $O_{2}$, оставаясь в одной плоскости и образуя шарнирный четырехзвенник.
Дано: длина стержня $O_{1}A=a$ и его угловая скорость $\omega$. Определить построением ту точку $М$ стержня
$AB$, скорость которой направлена вдоль этого стержня, а также найти величину скорости $v$ точки $M$ в тот момент,
когда угол $O_{1}AB$ имеет данную величину $\alpha$.
